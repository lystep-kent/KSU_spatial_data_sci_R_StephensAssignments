\documentclass[]{article}
\usepackage{lmodern}
\usepackage{amssymb,amsmath}
\usepackage{ifxetex,ifluatex}
\usepackage{fixltx2e} % provides \textsubscript
\ifnum 0\ifxetex 1\fi\ifluatex 1\fi=0 % if pdftex
  \usepackage[T1]{fontenc}
  \usepackage[utf8]{inputenc}
\else % if luatex or xelatex
  \ifxetex
    \usepackage{mathspec}
  \else
    \usepackage{fontspec}
  \fi
  \defaultfontfeatures{Ligatures=TeX,Scale=MatchLowercase}
\fi
% use upquote if available, for straight quotes in verbatim environments
\IfFileExists{upquote.sty}{\usepackage{upquote}}{}
% use microtype if available
\IfFileExists{microtype.sty}{%
\usepackage{microtype}
\UseMicrotypeSet[protrusion]{basicmath} % disable protrusion for tt fonts
}{}
\usepackage[margin=1in]{geometry}
\usepackage{hyperref}
\PassOptionsToPackage{usenames,dvipsnames}{color} % color is loaded by hyperref
\hypersetup{unicode=true,
            pdftitle={Lab 05},
            colorlinks=true,
            linkcolor=Maroon,
            citecolor=Blue,
            urlcolor=Blue,
            breaklinks=true}
\urlstyle{same}  % don't use monospace font for urls
\IfFileExists{parskip.sty}{%
\usepackage{parskip}
}{% else
\setlength{\parindent}{0pt}
\setlength{\parskip}{6pt plus 2pt minus 1pt}
}
\setlength{\emergencystretch}{3em}  % prevent overfull lines
\providecommand{\tightlist}{%
  \setlength{\itemsep}{0pt}\setlength{\parskip}{0pt}}
\setcounter{secnumdepth}{0}
% Redefines (sub)paragraphs to behave more like sections
\ifx\paragraph\undefined\else
\let\oldparagraph\paragraph
\renewcommand{\paragraph}[1]{\oldparagraph{#1}\mbox{}}
\fi
\ifx\subparagraph\undefined\else
\let\oldsubparagraph\subparagraph
\renewcommand{\subparagraph}[1]{\oldsubparagraph{#1}\mbox{}}
\fi

%%% Use protect on footnotes to avoid problems with footnotes in titles
\let\rmarkdownfootnote\footnote%
\def\footnote{\protect\rmarkdownfootnote}


  \title{Lab 05}
    \author{}
    \date{}
  

% change section title styling
\usepackage{sectsty}
\sectionfont{\normalsize\normalfont\itshape}
\subsectionfont{\normalsize\normalfont}

% use fancyhdr style
\usepackage{fancyhdr}
\pagestyle{fancy}
\fancyhead[LO, LE]{Lab 05}
\fancyhead[RO, RE]{Env Analysis in R}
\makeatletter
\renewcommand{\maketitle}{\bgroup\vspace*{-1cm}\setlength{\parindent}{0pt}
\begin{flushleft}
  \@author
  
  \@date
  
\end{flushleft}\egroup
}
\makeatother

\begin{document}
\maketitle

\section{Lab 05: Dynamic mapping in
Leaflet}\label{lab-05-dynamic-mapping-in-leaflet}

\subsubsection{Read the instructions COMPLETELY before starting the
lab}\label{read-the-instructions-completely-before-starting-the-lab}

This lab builds on many of the discussions and exercises from class,
including lab 4

\subsubsection{Formatting your
submission}\label{formatting-your-submission}

This lab must be placed into a public repository on GitHub
(www.github.com). Before the due date, submit \textbf{on Canvas} a link
to the repository. I will then download your repositories and run your
code. The code must be contained in either a .R script or a .Rmd
markdown document. As I need to run your code, any data you use in the
lab must be referenced using \textbf{relative path names}. Finally,
answers to questions I pose in this document must also be in the
repository at the time you submit your link to Canvas. They can be in a
separate text file, or if you decide to use an RMarkdown document, you
can answer them directly in the doc.

\subsection{Introduction}\label{introduction}

This lab is much more free-form than previous assignments. You will be
taking previous work from previous labs re-creating the maps using the
Leaflet package. I encourage you to use whatever resources you find
useful, including \url{https://rstudio.github.io/leaflet/} and
\url{https://bookdown.org/nicohahn/making_maps_with_r5/docs/leaflet.html}

\subsection{Your tasks}\label{your-tasks}

For this lab, you will re-create maps from labs 2-4 using Leaflet.
Below, I have referenced a specific question or task from lab 2, lab 3,
and lab 4. Further, I have added new, Leaflet-specific tasks to each
item. You will need to look back to previous labs to find the relevant
context for each task. Create the maps as specified, then answer the
questions at the end.

\subsubsection{Task 1. From lab 2, task
2.3:}\label{task-1.-from-lab-2-task-2.3}

\emph{Original task to be recreated using Leaflet:} Make a map of the
counties, shading each county by the total cost of BMPs
funded/implemented in that county. This will required you to join
multiple datasets together

\emph{Leaflet/other extras to add:}

\begin{itemize}
\item
  Mouse-over label the displays the total cost of BMPs funded in that
  county
\item
  Use an equal-interval classification method with 5 classes. Determine
  the breaks programmatically.
\item
  Do NOT use the default color scheme
\end{itemize}

\subsubsection{Task 2. From lab 3, task Bonus
\#2:}\label{task-2.-from-lab-3-task-bonus-2}

\emph{Original task to be recreated using Leaflet:} plot a choropleth
map of your dataset with a categorical color scheme, where the shading
corresponds to the Moran plot (really, ``LISA'') quadrants. Thus, your
map will have four shades of color.

\emph{Leaflet/other extras to add:}

\begin{itemize}
\item
  Add a pop-up window that displays the p-value (you'll have to look at
  the \texttt{moran.plot()} documentation) when you click on that county
  with a mouse
\item
  Add a control to change between 3 different basemaps
\end{itemize}

\subsubsection{Task 3: From lab 4, task
2:}\label{task-3-from-lab-4-task-2}

\emph{Original task to be recreated using Leaflet:} Make a second map of
your choosing. You may choose any spatial extent, domain, or technique.
I'm looking for creativity, good coding practices (including comments),
and for you to demonstrate independent thinking. There are minor
restrictions you must follow:

\begin{enumerate}
\def\labelenumi{\arabic{enumi}.}
\item
  It must include vector AND raster data in some manner
\item
  It must include spatial data relating to a social process (e.g.,
  political boundaries) AND spatial data relating to an environmental
  process (e.g., water resources)
\item
  The map should ``stand on its own'' and communicate its purpose
  without additional text
\item
  That's it!
\end{enumerate}

\emph{Leaflet/other extras to add:}

\begin{itemize}
\item
  Add a control that turns on/off each layer
\item
  Since everyone's maps are different, I can't specify exactly what else
  you should add. But, find one thing cool, interesting, or applicable
  to YOUR map, and implement it.
\end{itemize}

\subsection{\texorpdfstring{\textbf{Questions:}}{Questions:}}\label{questions}

\begin{enumerate}
\def\labelenumi{\arabic{enumi}.}
\item
  Reflect on the labs from this semester. What did you learn? What did
  you like? What did you not like?
\item
  Describe the ``one thing'' you chose to add to your map in Task 3
  above. What did you do, and why is it applicable to your map?
\end{enumerate}


\end{document}