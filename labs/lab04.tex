\documentclass[]{article}
\usepackage{lmodern}
\usepackage{amssymb,amsmath}
\usepackage{ifxetex,ifluatex}
\usepackage{fixltx2e} % provides \textsubscript
\ifnum 0\ifxetex 1\fi\ifluatex 1\fi=0 % if pdftex
  \usepackage[T1]{fontenc}
  \usepackage[utf8]{inputenc}
\else % if luatex or xelatex
  \ifxetex
    \usepackage{mathspec}
  \else
    \usepackage{fontspec}
  \fi
  \defaultfontfeatures{Ligatures=TeX,Scale=MatchLowercase}
\fi
% use upquote if available, for straight quotes in verbatim environments
\IfFileExists{upquote.sty}{\usepackage{upquote}}{}
% use microtype if available
\IfFileExists{microtype.sty}{%
\usepackage{microtype}
\UseMicrotypeSet[protrusion]{basicmath} % disable protrusion for tt fonts
}{}
\usepackage[margin=1in]{geometry}
\usepackage{hyperref}
\PassOptionsToPackage{usenames,dvipsnames}{color} % color is loaded by hyperref
\hypersetup{unicode=true,
            pdftitle={Lab 04},
            colorlinks=true,
            linkcolor=Maroon,
            citecolor=Blue,
            urlcolor=Blue,
            breaklinks=true}
\urlstyle{same}  % don't use monospace font for urls
\IfFileExists{parskip.sty}{%
\usepackage{parskip}
}{% else
\setlength{\parindent}{0pt}
\setlength{\parskip}{6pt plus 2pt minus 1pt}
}
\setlength{\emergencystretch}{3em}  % prevent overfull lines
\providecommand{\tightlist}{%
  \setlength{\itemsep}{0pt}\setlength{\parskip}{0pt}}
\setcounter{secnumdepth}{0}
% Redefines (sub)paragraphs to behave more like sections
\ifx\paragraph\undefined\else
\let\oldparagraph\paragraph
\renewcommand{\paragraph}[1]{\oldparagraph{#1}\mbox{}}
\fi
\ifx\subparagraph\undefined\else
\let\oldsubparagraph\subparagraph
\renewcommand{\subparagraph}[1]{\oldsubparagraph{#1}\mbox{}}
\fi

%%% Use protect on footnotes to avoid problems with footnotes in titles
\let\rmarkdownfootnote\footnote%
\def\footnote{\protect\rmarkdownfootnote}


  \title{Lab 04}
    \author{}
    \date{}
  

% change section title styling
\usepackage{sectsty}
\sectionfont{\normalsize\normalfont\itshape}
\subsectionfont{\normalsize\normalfont}

% use fancyhdr style
\usepackage{fancyhdr}
\pagestyle{fancy}
\fancyhead[LO, LE]{Lab 04}
\fancyhead[RO, RE]{Env Analysis in R}
\makeatletter
\renewcommand{\maketitle}{\bgroup\vspace*{-1cm}\setlength{\parindent}{0pt}
\begin{flushleft}
  \@author
  
  \@date
  
\end{flushleft}\egroup
}
\makeatother

\begin{document}
\maketitle

\section{Lab 04: Making static maps}\label{lab-04-making-static-maps}

\subsubsection{Read the instructions COMPLETELY before starting the
lab}\label{read-the-instructions-completely-before-starting-the-lab}

This lab builds on many of the discussions and exercises from class,
including the ``frankenmap'' exercise from class.

\subsubsection{Formatting your
submission}\label{formatting-your-submission}

This lab must be placed into a public repository on GitHub
(www.github.com). Before the due date, submit \textbf{on Canvas} a link
to the repository. I will then download your repositories and run your
code. The code must be contained in either a .R script or a .Rmd
markdown document. As I need to run your code, any data you use in the
lab must be referenced using \textbf{relative path names}. Finally,
answers to questions I pose in this document must also be in the
repository at the time you submit your link to Canvas. They can be in a
separate text file, or if you decide to use an RMarkdown document, you
can answer them directly in the doc.

\subsection{Introduction}\label{introduction}

This lab is much more free-form than previous assignments. You will be
completing your own version of the in-class ``frankenmap'' using proper
cartographic principles. I encourage you to use whatever resources you
find useful, including the relevant sections of the Lovelace chapter and
online resources such as:
\url{https://mgimond.github.io/Spatial/good-map-making-tips.html}

\subsection{Your tasks}\label{your-tasks}

\begin{enumerate}
\def\labelenumi{\arabic{enumi}.}
\tightlist
\item
  Using the same descriptions as the in-class frankenmap, create a map
  that follows ``good'' cartographic principles. I have included the
  instructions below for reference. The data can be found in:
  \texttt{./data/static\_mapping/}
\end{enumerate}

Original description:

\subsubsection{Ohio scale}\label{ohio-scale}

\begin{enumerate}
\def\labelenumi{\arabic{enumi}.}
\tightlist
\item
  Ohio counties, symbolized (filled) by some variable of interest. You
  will need to use a tabular join (\texttt{left\_join()}, most likely)
  between spatial and tabular data
\item
  Borders symbolized using NON-DEFAULT symbols
\item
  A scale bar
\end{enumerate}

\subsubsection{Local scale}\label{local-scale}

\begin{enumerate}
\def\labelenumi{\arabic{enumi}.}
\tightlist
\item
  Municipal boundaries within Portage AND Summit counties, with labels
  for names (see \texttt{oh\_places.gpkg})
\item
  Parks within Portage AND Summit counties, symbolized using different
  shades of green according to the park TYPE
\item
  Linear water features (streams, rivers) in Portage AND Summit
  counties. Symbols should indicate which linear features intersect a
  park
\end{enumerate}

\subsection{Putting it together}\label{putting-it-together}

\begin{enumerate}
\def\labelenumi{\arabic{enumi}.}
\tightlist
\item
  Use the provided DEM to plot elevation behind a semi-transparent
  Portage and Summit counties
\item
  A north arrow
\item
  Code to make Group 1's code (all of Ohio) an inset into Group 2's
  Portage + Summit counties study area
\item
  A title
\end{enumerate}

\subsubsection{One NOTE:}\label{one-note}

\emph{EVERYTHING NEEDS DONE PROGRAMMATICALLY REFERENCING DATA LOCATIONS
IN THE REPO}

\begin{enumerate}
\def\labelenumi{\arabic{enumi}.}
\setcounter{enumi}{1}
\tightlist
\item
  Make a second static map of your choosing. You may choose any spatial
  extent, domain, or technique. I'm looking for creativity, good coding
  practices (including comments), and for you to demonstrate independent
  thinking. There are minor restrictions you must follow:

  \begin{enumerate}
  \def\labelenumii{\arabic{enumii}.}
  \tightlist
  \item
    It must include vector AND raster data in some manner
  \item
    It must include spatial data relating to a social process (e.g.,
    political boundaries) AND spatial data relating to an environmental
    process (e.g., water resources)
  \item
    The map should ``stand on its own'' and communicate its purpose
    without additional text
  \item
    That's it!
  \end{enumerate}
\end{enumerate}

\subsection{Questions:}\label{questions}

\begin{enumerate}
\def\labelenumi{\arabic{enumi}.}
\item
  Describe your choices in making map 1
\item
  Describe your choices in making map 2. Include why you chose the
  problem and where you obtained your data. Finally, your map is a
  communication piece. What was the intent of your communication and do
  you feel as though you achieved your goal?
\item
  What did you learn?
\end{enumerate}


\end{document}